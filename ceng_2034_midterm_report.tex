
\documentclass[onecolumn]{article}
%\usepackage{url}
%\usepackage{algorithmic}
\usepackage[a4paper]{geometry}
\usepackage{datetime}
\usepackage[margin=2em, font=small,labelfont=it]{caption}
\usepackage{graphicx}
\usepackage{mathpazo} % use palatino
\usepackage[scaled]{helvet} % helvetica
\usepackage{microtype}
\usepackage{amsmath}
\usepackage{subfigure}
% Letterspacing macros
\newcommand{\spacecaps}[1]{\textls[200]{\MakeUppercase{#1}}}
\newcommand{\spacesc}[1]{\textls[50]{\textsc{\MakeLowercase{#1}}}}

\title{\spacecaps{Assignment Report 1: Process and Thread Implementation}\\ \normalsize \spacesc{CENG2034, Operating Systems} }

\author{Murat Doğan\\muratdogan5@posta.mu.edu.tr}
%\date{\today\\\currenttime}
\date{\today}

\begin{document}
\maketitle




\section{Introduction}
This assignment which the given us, we see how to understand process and thread implementation.

\section{Assignments}
In this section, i will show what i do in this assignment with articles.

\subsection{ Print PID of itself. }

Using the os library, i use getpid() function to get pid.
    \newline
    \includegraphics[width=\textwidth]{fig/assignment1.PNG}
\subsection{ If the running OS is linux; print loadavg.  }

Using the os library again, i used getloadavg() function to get load average if my operating system is linux
    \newline
    \includegraphics[width=\textwidth]{fig/assignment2.PNG}

\subsection{  Take and print “5 min loadavg” value and cpu core count. If the loadavg value is near (or close) to the cpu core count (hint: nproc - 5min loadavg < 1) then exit script.  }
cpu\_count give us the cpu core count. i put getloadavg data in a list and second element of array give us 5 min loadavg. there is a if statement included if cpu core count minus 5 min loadavg less than one exit the script. otherwise, print the message.
    \newline
    \includegraphics[width=\textwidth]{fig/assignment3.PNG}

\subsection{ Check if the links in these urls are valid (working) or not. (Hint: You can use python requests lib. HTTP response status code 2xx is successful. 4xx or 5xx means failed.) *Implement this with threads }

i create a function called by check. it checks a link is valid or not and it works by threads
    \newline
    \includegraphics[width=\textwidth]{fig/assignment4.PNG}




\section{Results}
in result, this script gives us, getting pit, loadavg and a link is valid or not.





\section{Conclusion}
in conclusion i used os, threading, requests library in python.

\section{GitHub}
github.com/dognmrt/ceng\_2034\_2020\_midterm

\end{document}

